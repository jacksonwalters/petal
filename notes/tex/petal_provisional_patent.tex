% !TEX TS-program = pdflatex
% !TEX encoding = UTF-8 Unicode

% This is a simple template for a LaTeX document using the "article" class.
% See "book", "report", "letter" for other types of document.

\documentclass[11pt]{article} % use larger type; default would be 10pt

\usepackage[utf8]{inputenc} % set input encoding (not needed with XeLaTeX)

%%% Examples of Article customizations
% These packages are optional, depending whether you want the features they provide.
% See the LaTeX Companion or other references for full information.

%%% PAGE DIMENSIONS
\usepackage{geometry} % to change the page dimensions
\geometry{a4paper} % or letterpaper (US) or a5paper or....
% \geometry{margin=2in} % for example, change the margins to 2 inches all round
% \geometry{landscape} % set up the page for landscape
%   read geometry.pdf for detailed page layout information

\usepackage{graphicx} % support the \includegraphics command and options

% \usepackage[parfill]{parskip} % Activate to begin paragraphs with an empty line rather than an indent

%%% PACKAGES
\usepackage{booktabs} % for much better looking tables
\usepackage{array} % for better arrays (eg matrices) in maths
\usepackage{paralist} % very flexible & customisable lists (eg. enumerate/itemize, etc.)
\usepackage{verbatim} % adds environment for commenting out blocks of text & for better verbatim
\usepackage{subfig} % make it possible to include more than one captioned figure/table in a single float
\usepackage{amsmath}
\usepackage{amssymb}
% These packages are all incorporated in the memoir class to one degree or another...

%%% HEADERS & FOOTERS
\usepackage{fancyhdr} % This should be set AFTER setting up the page geometry
\pagestyle{fancy} % options: empty , plain , fancy
\renewcommand{\headrulewidth}{0pt} % customise the layout...
\lhead{}\chead{}\rhead{}
\lfoot{}\cfoot{\thepage}\rfoot{}

%%% SECTION TITLE APPEARANCE
\usepackage{sectsty}
\allsectionsfont{\sffamily\mdseries\upshape} % (See the fntguide.pdf for font help)
% (This matches ConTeXt defaults)

%%% ToC (table of contents) APPEARANCE
\usepackage[nottoc,notlof,notlot]{tocbibind} % Put the bibliography in the ToC
\usepackage[titles,subfigure]{tocloft} % Alter the style of the Table of Contents
\renewcommand{\cftsecfont}{\rmfamily\mdseries\upshape}
\renewcommand{\cftsecpagefont}{\rmfamily\mdseries\upshape} % No bold!

%%% END Article customizations

%%% The "real" document content comes below...

\title{Provisional Patent Application for Petal}
\author{Jackson Walters}
%\date{} % Activate to display a given date or no date (if empty),
         % otherwise the current date is printed 

\begin{document}
\maketitle

\section{Notes}

Petal - DIY vintage bottom bracket integrated, sensorless, brushless, DC 100-250w 6 pole stator, 4 pole rotor, mid-drive eBike motor



Weds. 1/15/2020 (update 5/2020): Round two at AA a success - got the 500w motor running at low power, ~6w. Step 1 complete! Here’s the plan:

take apart existing motor, understand how it works - DONE
hook up oscilloscope, understand controller better - DONE

learn to 3d print (tool training 3/22) - DONE 
3d print stator model for old school bottom bracket - DONE

learn to lost wax/investment cast/machine/CNC to produce metal/iron stator core
embed/attach permanent magnets to BB spindle - DONE
shave down spindle and use bigger, stronger magnets
wind stator - DONE

bench power supply, get spinning - DONE
48v e-bike battery
cut bike frame to tabletop size
install BB motor + controller + battery

test w/ cranks, *torque*

install prototype e-motor on full bike
test prototype



make a few working prototypes
hand out to others for testing/alpha

make video documenting build process
make promo video for web

launch website w/ video(s)
crowdfunding/beta

solicit investors
patent
form llc/startup

refine design
contract manufacturer
set up direct sale on website

release v1


vintage bottom bracket w/ spindle, bearings, and cups
48v battery
controller/driver
4 strong, curved magnets
jb-weld epoxy
stator iron (3d print mold, then lost-wax cast)
copper wire

SOLVED: nylon/metal washer possibly with notches for bearing cups to compress stator. or just design the stator so that it juts out enough for the bearings to compress on. need so that stator stays centered and fixed (non-rotating)

use cable guide screw hole for wires 

https://artisansasylum.com/




Weds. 1/22/2020: heading to AA tonight to hook up motor to scope, possibly take apart 500w motor.

Fri. 5/1/2020: 3d printed the stator, v2. This print was successful.

resources:
printer = Prusa i3 MK3S
material = prusament PLA (galaxy purple)
amount = 16.64g
cost = \$.42
time = 2h12min
settings:
slicer = PrusaSlicer
layer height = .20mm quality
infill = \%15
supports = none




v2 dimensions:
\begin{enumerate}
\item $num poles = 6$
\item td = total diameter - 38mm (=(1+pr)*od)
\item od = center ring outer diameter - 21mm
\item id = center ring inner diameter - ~17mm
\item h = height - 43mm
\item pr = pole ratio - .65 => pole length ~ 6.5mm (= pr*(od/2) )
\end{enumerate}



Pretty damn close to the specifications for the generated object file.

For v3, I’ll need to have the inner diameter wide enough to fit over the races of the spindle.

vintage bottom bracket dimensions:
center length = 43.5mm (bottom groove to bottom groove. ~45mm race to race)
spindle diameter = 16.5mm
square taper (small) = 13mm
square taper (large) = 14mm
race diameter = 21mm

standard bottom tube dimensions:
inner diameter = 34mm



v3 dimensions:
\begin{enumerate}
\item $num_poles = 6$
\item td = total diameter - 34mm
\item od = center ring outer diameter - 21mm
\item id = center ring inner diameter - ~17mm
\item h = height - 43mm
\item pr = pole ratio - .65 => pole length ~ 6.5mm (= pr*(od/2) )
\end{enumerate}

~5:45pm: In progress.

Printed successfully. With the inner diameter exactly matching the bearing race diameter, it was remarkably tight. With a lot of force and some time, I was able to get it past. It may make sense to leave it that small with a little internal sanding or something, but I’d prefer .1mm on each side and just have an inner diameter that’s .2mm bigger.



Sun. 5/10/20, ~8:30pm:

Trying v4. This time, we have:

\begin{enumerate}
\item $pole_num=6$
\item $id=21.2mm$
\item $od=25.2mm$
\item $height=43mm$
\item $pol_ratio=.25$
\item $cap_ratio=1.5$
\item $resolution=400$
\end{enumerate}

I noticed that a problem I thought I have is not a problem. When I go to wrap the copper windings, it will bulge out a little on the sides, 1-3mm. I thought I might have to shave the stator height from 43mm to 40mm to compensate. However, there is a nice 3-4mm gap on either side between the bearing cage and the end of the cup.





Printing v4. Added a brim, turned the $bed_temp$ up to 70c, and turned the speed down on the first few layers. Making sure to have good adhesion.

v4 printing successfully, will take 1h3m, 15.95g. PLA @ $24.99/kg = $0.40.

Mon. 5/11/20: Realized that, with the bearing cups and cages, the available length for the stator ‘height’ is really 42mm, not 43mm. It may be that you can compress the stator, but probably only .2mm or so, which would allow for a nice snug fit.

Ordered some neodymium arc magnets from https://www.apexmagnets.com/.

v5 has height 42.1mm.

~7pm: v5 finished printing. It’s still a little too big at 42.1mm. Will try 41.1mm.




V4 is printed. It measures:

\begin{enumerate}
\item $total_diam = 34.5mm$
\item $id = ~21.2mm$
\item $od =  ~25.1mm$
\item $pole_length = 2.9mm$
\end{enumerate}


Calculations:

Would like to calculate power. Need to know:

\begin{enumerate}
\item $wire gauge$
\item $number of windings$
\item $current$
\item $strength of magnets$
\item $geometry of setup$
\item $number of poles = 6$
\end{enumerate}

$$pole_length = 2.9mm$$
$$pole_width ~ 5mm$$

$$AWG \rightarrow diam \rightarrow num_windings:$$

\begin{enumerate}
\item $19 (1.8amp) \rightarrow .912mm \rightarrow 3$
\item $20 (1.5amp) \rightarrow .812mm \rightarrow 3$
\item $21 (1.2amp) \rightarrow .723 \rightarrow 4$
\item $22 (.92) = .644 \rightarrow 4$
\end{enumerate}

Example: AWG 21, 1.2amp max power transmission, .723mm, 4 windings around pole, 6 layers.

24 loops per pole.

Too hard - need to measure. At 48v, no more than 57watt.

\subsection{Tues. 5/12/20:} Printing v7, height = 41.3mm. v6 fit perfectly, but not snugly. need some compression.

\subsection{Thurs. 5/14/20:} Magnets arrived yesterday. At least one set fits very nicely, no need to shave down the spindle yet, and they stick right on so no immediate need for epoxy. Going to grab some copper wire, and eventually will need to find a 48v battery.





\subsection{~10pm, Thurs. 5/14/20:} Found some magnet wire at “You-Do-It” Electronics Center in Needham. 22 gauge. Bought a multimeter, cutters, and wire strippers at Home Depot. Wound the stator core, and was able to generate a few mA of current by spinning the spindle.




\subsection{5/16/20:} Got a nice 30V, 5A variable Korad power supply for testing. Got 500w brushless motor spinning at 25W.



\subsection{~6:32pm, 5/19/20:} Got some precision files to file down the notches near the end of the stator poles a bit more to better store the windings. They weren’t quite fitting into the gap in the bearing cups between the bearing cages and the stator itself. 

First test w/ power, spinning: https://www.youtube.com/watch?v=SrjFXLMFxuI
Spinning, brief: https://youtu.be/y2wRbTZY77E
500w motor w/ Korad, test: https://youtu.be/1-twHIf9ygM
Generating power w/ multimeter: https://youtu.be/IRpmhwM5PVo
Spinning by hand: https://youtu.be/fcTMyOD5his




The motor is spinning when hooked up to the 3-phase controller. Without mounting to a cut frame or the like, the axle causes the bearing cups to spin off. There may be a pair of windings wired in the wrong direction as it’s doing some back and forth spinning. There are also no Hall sensors, which may be part of it. It is most likely just the vibration.

The bearing cups rubbing on the enamel coated magnet wire causes it to rub off a little and conduct on the metal, causing sparking. This can easily be alleviated by just filing the notches a bit more so that the stator completely encases the windings.


\end{document}
